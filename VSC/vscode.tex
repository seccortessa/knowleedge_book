\chapter{Visual Studio Code}

\section{Snippets personalizados}

Todos los archivos de snippets están escritos en formato JSON. Existen archivos propios para cada lenguage y existen también para uso global en VSCode. La sintaxis para configurar snippets personales es la que se muestra a continuación.

\begin{verbatim}
"title": {
    "prefix": "chapter",
    "body": "\\chapter{$1}"
}
\end{verbatim}

Los items más importantes son \texttt{"prefix"}, que indica el texto que se teclea para que intellisense reconozca el snippet y lo ejecute. El segundo es \texttt{"body"}, que indica el texto que se usa dentro del snippet. Las directivas \texttt{$1,$2,...} son usadas para ubicar el cursor para llenar las opciones dentro del snippet. También pueden tener identificadores con palabras y no solamente números.\\

Para proponer una porción de código de varias líneas, se puede usar la siguiente sintaxis

\begin{verbatim}
"Add a image": {
    "prefix": "figure",
    "body": [
        "\\begin{figure}[$1]",
        "\t\\centering",
        "\t\\includegraphics[width=$2\\columnwidth]{$3}",
        "\t\\caption{$4}",
        "\t\\label{$5}",
        "\\end{figure}",
    ],
    "description": "Add an image"
\end{verbatim} 

Se puede observar también la directiva \texttt{\textbackslash t} para proponer dentro del snippet la inclusión de identación.


\section{Edicion de texto con markdown para la documentación}


Una herramienta que puede ser útil para los repositorios es la de edición de la documentación con markdown, que con la opción \texttt{Markdown: Preview} permite ver cómo está quedando la documentación, y con la opción \texttt{Markdown: open preview on the side} para que se actualice en tiempo real.
\begin{itemize}
    \item Título principal: \texttt{\# Title} \\
    \item Título secundario: \texttt{\#\# Subtitle} \\
    \item Título de sub-sección: \texttt{\#\#\# Subsection} \\
    \item Numeración normal: \texttt{1. text}
    \item Adición de texto en formato código: \texttt{See the following code : 'sudo apt install -g typescript'} \\
    \item Inserción de una URL a una palabra en particular: \texttt{You can see the documentation in this [link](https://github.com/) } \\
\end{itemize}

Para la creación de tablas con Markdown se utilizan la barra vertical:


\begin{verbatim}
| Position | Url |
|----------|-----|
|All employees  | localshost |
\end{verbatim}








\section{Organizacion del código}

Una herramienta útil para el uso del código es la de reemplazar todas las ocurrencias para una palabra, con el comando \texttt{Ctrl + F2} se renombra todas las palabras realcionadas. Otro comando importante es \texttt{Ctrl+D} el cual selecciona solamente la palabra, en caso de querer reemplazarla o simplemente borrarla.


En ocasiones es necesario tener varias terminales para diferentes propósitos dentro del proyecto, una configuración útil para mejorar la productividad es la siguiente configuración de teclado:


\begin{verbatim}
{
    "key": "ctrl+tab",
    "command": "workbench.action.terminal.focusNext",
    "when": "terminalFocus"
}

{
    "key": "ctrl+shift+tab",
    "command": "workbench.action.terminal.focusPrevious",
    "when": "terminalFocus"
}

{
    "key": "ctrl+n",
    "command": "workbench.action.terminal.new",
    "when": "terminalFocus"
}

{
    "key": "ctrl+w",
    "command": "workbench.action.terminal.kill",
    "when": "terminalFocus"
}

{
    "key": "ctrl+r",
    "command": "workbench.action.terminal.rename",
    "when": "terminalFocus" 
}
\end{verbatim}

en el archivo \texttt{Json} de los atajos de teclado.

\section{Definición de 'workspace'}

Los entornos de trabajo son básicamente una colección de carpetas las cuales podemos manejar en una misma ventana de VSCode. Estos se guardan con la extensión \texttt{.code-workspace}. Cada espacio de trabajo se puede ver como una entidad diferente en la cual estoy realizando el trabajo de código.


\section{Notas adicionales}

\subsection{Comandos importantes}

La siguiente es una lista importante de los comandos que facilitan la productividad



\begin{itemize}
    \item \texttt{Alt + Up/Down} Mueve la línea o líneas de código hacia arriba o hacia abajo 
    \item \texttt{Shift + Alt + T} Muestra u oculta el terminal de VSCode
    \item \texttt{Ctrl + L} Selecciona toda la línea
    \item \texttt{Ctrl + D} Selecciona la palabra actual
    \item \texttt{Ctrl + F3 / Shift+Ctrl+F3} Cuando el cursor está sobre una palabra, selecciona la palabra y mueve el cursor hacia la siguiente lalabara duplicada.
    \item \texttt{Shift+Alt+K} Borra la línea completa
    \item \texttt{Ctrl+Shift+E} Abre el explorador de la parte izquierda
    \item \texttt{Ctrl+Shift+L} Selecciona todas las palabras repetidas
    \item \texttt{Ctrl+Shift+M} Abre o cierra el panel de problemas de la parte inferior
    \item \texttt{Ctrl+Shift+Y} Abre o cierra la consola de debug de la parte inferior
    \item \texttt{Ctrl+Shift+U} Abre o cierra el panel de salida de la parte inferior
    \item \texttt{Ctrl+Enter} Hace un salto de línea si importar en qué parte de lña línea se encuentre
    \item \texttt{Shift+Enter} Ejecuta el script
    \item \texttt{Ctrl+Tab} CAmbia entre pestañas abiertas
    \item \texttt{Shift+Alt+A} Comentar/quitar comentario de un bloque de código seleccionado
    \item \texttt{Ctrl+\}} Comentar/quitar comentario de una línea
    Si en el buscador al que accedemos con \texttt{Ctrl+Shift+P} ponemos dos puntos seguidos con un número, automáticamente se va a seleccionar esa línea de código.
    \item \texttt{Ctrl+B} Muestra o quita la barra de la izquierda (Explorador,Extensiones, etc)
    \item \texttt{F2} al tener una palabra seleccionada, si se trata de una dofinición de clase, o una palabra repetida, sirve para cambiar esa definición y se cambiarán todas las ocurrencias de esa palabra.
    \item \texttt{Shift+Alt+Up/Down} coia la línea entera de código en la línea siguiente o en la linea anterior.
\end{itemize}

\subsection{Concepto de tareas}

 

Las tareas en VSCode son una herramienta que sirve para automatizar procesos dentro de VSCode. La creación y ejecución de tareas está manejado mediante archivos JSON. En estos se configura la lista de tareas que serán ejecutadas cuando la tarea asea ejecutada.


\subsection{Debug} 
También es posible personalizar una configuración JSON para ejecutar el programa con el depurador.